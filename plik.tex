\documentclass[12pt]{article}
\usepackage[utf8]{inputenc}
\usepackage{graphicx}
\usepackage{amsmath}
\usepackage{gensymb}
\graphicspath{ {c:/limage/} }
\DeclareGraphicsExtensions{.png}
\usepackage[hmargin={1.6 cm,1.5cm},
   top=1.5cm, marginpar=3.5cm
   ]{geometry}
\usepackage{polski}
\author{https://pl.wikipedia.org/wiki/Planeta{\_}X}

\title{\textsf{Planeta X}}
\begin{document}
\maketitle
\section{"Spis treści"}
\begin{itemize}
  \item O planecie \\
\begin{figure}[!ht]
\includegraphics[width=3cm,height=3cm,keepaspectratio]{1.png}\caption{Wizja artystyczna hipotetycznej dziewiątej planety zbliżonej budową do Neptuna}\label{fig:1}
\end{figure}
  \item Kontekst historyczny \\
\begin{figure}[!ht]
\includegraphics[width=3cm,height=3cm,keepaspectratio]{2.png}\caption{Percival Lowell przy pracy w ufundowanym przez siebie obserwatorium}\label{fig:2}
\end{figure}
  \item Współczesny stan wiedzy \\
  \item Klif Kuipera  \\
\begin{figure}[!ht]
\includegraphics[width=3cm,height=3cm,keepaspectratio]{3.png}\caption{Liczba znanych ciał pasa Kuipera w zależności od odległości od Słońca}\label{fig:3}
\end{figure}
  \item Wewnętrzny Obłok Oorta \\
  \item Źródło komet długookresowych  \\
\begin{figure}[!ht]
\includegraphics[width=3cm,height=3cm,keepaspectratio]{4.png}\caption{Orbita hipotetycznej planety i orbity sześciu planetoid, które zostały zmienione po jej bliskim przejściu}\label{fig:4}
\end{figure}
  \item „Planet Nine” \\
\begin{figure}[!ht]
\includegraphics[width=3cm,height=3cm,keepaspectratio]{5.png}\caption{Jeżeli istnieje, to dziewiąta planeta jest prawdopodobnie lodowym olbrzymem o budowie podobnej do Urana i Neptuna (na ilustracji niebieskie planety dla porównania pokazano w jednakowej skali największe gazowe olbrzymy i planety skaliste)}\label{fig:5}
\end{figure}
\ldots
\end{itemize}
\newpage

\section{O Planecie}
{\bf Planeta X} - hipotetyczna dziewiąta planeta w Układzie Słonecznym, znajdująca się dalej od Słońca niż Neptun.\\
Rysunek \ref{fig:1}\\
W przeszłości istnienie takiej planety postulowano na podstawie nieprawidłowo wyliczonej perturbacji orbity Neptuna, co jednak doprowadziło do odkrycia Plutona. Pluton nie był odpowiedzialny za perturbacje orbity Neptuna, ale do czasu zmiany jego statusu z planety na planetę karłowatą był uznawany za dziewiątą planetę\\

W świetle obecnego stanu wiedzy, jeżeli w Układzie Słonecznym istnieje dziewiąta planeta, to musi krążyć znacznie dalej od Słońca, poza Pasem Kuipera, i nie wywierać mierzalnego wpływu na znane planety, ale może mieć znaczący wpływ na orbity obiektów transneptunowych. W 2015 rozpoczęto poszukiwania hipotetycznej dziewiątej planety po opublikowaniu teoretycznych wyliczeń wskazujących na możliwość jej istnienia.\\

W 1781 William Herschel odkrył nieznaną wówczas planetę – Urana. 40 lat później francuski astronom, Alexis Bouvard, opublikował tablice astronomiczne zawierające obliczone położenia planety. Obserwacje jednak nie zgadzały się z jego obliczeniami, co doprowadziło go do wniosku, że ruch Urana jest zaburzany przez oddziaływanie z inną planetą. Urbain Le Verrier i John Couch Adams niezależnie wyznaczyli, w którym miejscu na niebie powinno znajdować się to nieznane ciało. W 1846 Johann Gottfried Galle skierował teleskop w punkt wskazany przez Le Verriera i tuż obok niego odkrył Neptuna.\\
\section{Kontekst historyczny}
Rysunek \ref{fig:2}\\
Niedługo później różni uczeni (jako pierwszy Jacques Babinet) doszli do wniosku, że ruch Neptuna także jest zaburzany i uznano, że w większej odległości od Słońca znajduje się jeszcze co najmniej jedna planeta. Babinet, który w porównaniu z Le Verrierem był co najwyżej amatorskim astronomem, oskarżył tego ostatniego o wiele „potężnych błędów”, które wkradły się w wyliczenia dotyczące orbity i masy Neptuna. Babinet zaproponował przy tym istnienie innej planety, leżącej poza orbitą Neptuna, którą nazwał „Hyperion” i określił jej masę na 11,6 razy większą od masy Ziemi, a odległość od Słońca na 47–48 j.a.. Atak Babineta na Le Verriera był bardzo gwałtowny i osobisty; w obronie Le Verriera stanęli między innymi John Herschel, Wilhelm Struve i Carl Jacobi. Le Verrier bez trudności odparł zarzuty Babineta, niemniej gwałtowne i niepotrzebne wejście Babineta do dyskusji na temat Neptuna i hipotetycznej planety leżącej poza nim na długo zatruło atmosferę w ówczesnym środowisku astronomicznym i było bardzo typowe dla tej epoki.\\

Wielu XIX-wiecznych astronomów, w tym Le Verrier, sądziło, że za Neptunem istnieje jeszcze jakaś planeta, ale Le Verrier uważał, że jej poszukiwania nie mają sensu, zanim nie zdobędzie się kilku dekad danych obserwacyjnych dotyczących parametrów orbitalnych Neptuna, i zwrócił swoją uwagę ku Merkuremu. Uczonym, który rozpoczął systematyczne poszukiwania nieznanej planety zaburzającej ruchy Urana i Neptuna, był Percival Lowell. Zaproponował dla niej nazwę „Planeta X”. Niemniej sam Lowell zdawał sobie sprawę z trudności związanych z poszukiwaniem nowej planety, jak napisał w publikacji Memoir on a Trans-Neptunian Planet: „nie możemy użyć Neptuna jako drogowskazu do następnej planety, tak samo jak użyliśmy Urana do wskazania Neptuna, ponieważ nie mamy wystarczająco dużo obserwacji związanych z Neptunem”.\\

Jednym z uczonych poszukujących Planety X był Clyde Tombaugh. Jego obserwacje doprowadziły do odkrycia Plutona w 1930. Wkrótce jednak okazało się, że Pluton jest zbyt mały, żeby wywoływać zauważalne perturbacje ruchu planet. Kolejne dziesięciolecia obserwacji nie przyniosły rozwiązania problemu.\\

Dopiero badania sondy Voyager 2 pozwoliły rozwiązać zagadkę zaburzeń. Okazało się, że masa Neptuna została przeszacowana o ok. 0,5%. Uwzględnienie tej różnicy sprawiło, że zaburzenia zniknęły, zatem Planeta X, tak jak ją zdefiniował Lowell, nie istnieje.\\

\section{Współczesny stan wiedzy}
Ruch znanych planet nie jest zaburzany przez duży, nieznany obiekt, zatem pierwotna hipoteza o istnieniu Planety X jest fałszywa. Pewne fakty obserwacyjne mogą wskazywać jednak, że w Układzie Słonecznym istnieje nieznane ciało planetarne krążące po bardzo oddalonej od Słońca orbicie.\\

\subsection{Klif Kuipera}
W latach 90. XX wieku okazało się, że odkryty 60 lat wcześniej Pluton należy do większej grupy obiektów transneptunowych, które tworzą zewnętrzny pas planetoid skalno-lodowych – pas Kuipera – oraz dysk rozproszony. W grupie tej są także inne ciała dostatecznie masywne, aby pod wpływem własnej grawitacji utrzymać kształt bliski kulistemu (tzw. plutoidy), w tym Eris o masie większej od Plutona.\\
Rysunek \ref{fig:3}\\
Astronomowie znają coraz więcej obiektów pasa Kuipera, ale okazuje się, że w odległości ok. 50 j.a. od Słońca ich liczba drastycznie spada – jest to tzw. klif Kuipera. Takie ostre granice nie powstają bez powodu; przykładowo przerwy Kirkwooda w pasie planetoid tworzy grawitacja krążącego dalej od Słońca Jowisza. Prawdopodobnym wyjaśnieniem genezy klifu Kuipera jest oddziaływanie grawitacyjne z nieznanym obecnie ciałem o dużej masie. Jeden z modeli komputerowych dynamiki Układu Słonecznego, który stworzyli Patryk Lykawka i Tadashi Mukai z Uniwersytetu Kobe, wskazuje, że w odległości 100–170 j.a. od Słońca krąży planeta o masie ok. 30–70% masy Ziemi. Może ona mieć średnicę 10–15 tys. km i obiegać Słońce w czasie od 1000 do 2500 lat, po orbicie nachylonej do ekliptyki pod kątem 20–40°.\\

\subsection{Wewnętrzny Obłok Oorta}
W 2003 odkryta została duża planetoida (90377) Sedna krążąca po ekscentrycznej orbicie z peryhelium poza pasem Kuipera. Na jej orbitę nie ma już istotnego wpływu oddziaływanie Neptuna; jest to tzw. obiekt odłączony, przedstawicielka wewnętrznego Obłoku Oorta. W 2012 odkryto obiekt 2012 VP113 krążący po podobnej orbicie; może to wskazywać, że orbity obu tych ciał są kształtowane przez oddziaływanie nieznanego obiektu o większej masie, przypuszczalnie tego samego, które odpowiada za istnienie klifu Kuipera. Według odkrywców może to być nieznana planeta typu superziemia.\\

\subsection{Źródło komet długookresowych}
Inna hipoteza sugeruje, że komety długookresowe przybywają w pobliże Słońca z odległego Obłoku Oorta nie równomiernie, ale z pewnego pasa na niebie. Mogłoby to wskazywać, że są one wytrącane z pierwotnych orbit przez grawitację niezaobserwowanej dotąd masywnej planety. Obiekt ten, nazwany przez postulatorów „Tyche”, miałby być gazowym olbrzymem o masie 1–4 mas Jowisza, który krąży w odległości 10 000–30 000 j.a. w zewnętrznym Obłoku Oorta.\\
 Rysunek \ref{fig:4}\\
\subsection{„Planet Nine”}
W 2016 Konstantin Batygin i Michael E. Brown ogłosili wyniki badań, według których istnieją bardzo silne przesłanki teoretyczne na istnienie bardzo odległej, masywnej planety. Nie odkryli planety bezpośrednio, ale wywnioskowali jej potencjalne istnienie na podstawie symulacji komputerowych. Według astronomów planeta, nazywana przez nich po prostu „Planet Nine” (Dziewiąta Planeta), miałaby mieć masę około 20 mas Ziemi i przeciętnie być oddalona od Słońca dwadzieścia razy dalej niż Neptun. Według ogłoszonych wyników badań „Planet Nine” miałaby być odpowiedzialna między innymi za nietypowe, prostopadłe do ekliptyki orbity sześciu obiektów transneptunowych i kształty ich orbit, które ukierunkowane są w jedno miejsce.
\begin{table}[h]
	\begin{center}
		\begin{tabular}{|c|c|c|c|c|} \hline
			Obiekt & Okres orbitalny(lata) & Półos wielka(j.a.) & Peryhelium(j.a.) & Argument perycentrum($\omega)\\ \hline
			2010 GB_1_7_4 & 6855 & 361 & 48,5 & 347,3\degree\\ \hline
			2004 VN_1_1_2 & 5845 30 & 324,5 & 47,3312 & 327,20\degree\\ \hline
			2013 RF_9_8 & 5860 & 325 & 36,29 & 316,5\degree\\ \hline
			(90377) Sedna & 11400 & 524,4 & 76,0917 & 311,29\degree\\ \hline
			2012 VP_1_1_3 & 4268 179 & 263 & 80,5 & 294\degree\\ \hline
			2007 TG_4_2_2 & 11200 & 501 & 35,560 & 285,6\degree\\ \hline
		\end{tabular}
		\caption{Wyniki badań}
		\label{tab:1}
	\end{center}
\end{table}

Rysunek \ref{fig:5}\\

Jeżeli planeta rzeczywiście istnieje, to może być to hipotetyczny piąty olbrzym, którego istnienie sugerowane jest przez tzw. model nicejski opisujący migrację masywnych planet (olbrzymów – gazowych i lodowych) z ich oryginalnych orbit położonych znacznie bliżej Słońca do ich obecnych orbit. Jeśli istnieje, planeta ma bardzo wydłużoną, eliptyczną orbitę, a jej okres orbitalny wynosi 10–20 tysięcy ziemskich lat. Średnia odległość od Słońca hipotetycznej planety wynosiłaby 600 jednostek astronomicznych, ale mogłaby się zbliżać do Słońca nawet do 200 j.a..\\

Badania Batygina i Browna zostały ogłoszone w „The Astronomical Journal”. Inni naukowcy specjalizujący się w badaniach nad Pasem Kuipera, w tym jego odkrywca David Jewitt, zwracają uwagę na statystyczne problemy związane z interpretacją danych Batygina i Browna. Wyliczone prawdopodobieństwo takiego ustawienia sześciu planetoid, jak zostało zaobserwowane, wynosi około 0,007%, co daje odchylenie standardowe wynoszące około 3,8. Jest to więcej niż zazwyczaj wymagany poziom trzech odchyleń standardowych (99,7%), aby dane odkrycie zostało potraktowane poważnie, ale nadal jest to jedynie hipoteza. Inne analizy statystyczne sugerują, że obiekty pasa Kuipera mogą oddziaływać na siebie wzajemnie w taki sposób, aby stworzyć właśnie takie małe systemy planetoid.\\

Pomimo wątpliwości co do ogłoszonych wyników badań rozpoczęto już poszukiwania hipotetycznej planety. Oczekuje się, że obserwowana wielkość gwiazdowa takiego obiektu będzie wynosiła mniej niż 22, co czyniłoby ją przynajmniej 600 razy ciemniejszą od Plutona. Głównym teleskopem zajmującym się poszukiwaniem hipotetycznej planety jest japoński Subaru. Poszukiwania będą też przeprowadzane przy pomocy 10-metrowego teleskopu Kecka na Hawajach.\\

\begin{thebibliography}{9}

\bibitem{zrodlo}
	https://pl.wikipedia.org/wiki/Planeta_X
\end{thebibliography}
\end{document}